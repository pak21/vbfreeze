\documentclass[a4paper,12pt]{article}
\pagestyle{empty}
\usepackage{hyperref}
\usepackage{amsmath}

\begin{document}

\begin{center}
  {\LARGE \textbf{Volleyball's ``freeze'' proposal}}
\end{center}

There is a proposal being made[1] to change how (beach) volleyball's scoring
would work at the end of a set: the short version of it is that a team would
have to win a point while serving to win a set, rather than being able to win
by scoring a point on the opponent's serve. This is significant as the majority
of points in both beach and indoor volleyball are won by the receiving team.

\section{Assumptions}

In order to model this, we make a couple of assumptions:

\begin{itemize}
  \item That the chance of each team winning a point when serving is equal.
  \item That the chance of a team winning a point when serving is independent
    of the match situation.
\end{itemize}

The first assumption seems reasonable for a close match, which is all we really
care about; the second assumption may need validating, but we know from other
sports (particularly baseball and basketball) that ``clutch'' performance is
largely a myth so I think it's a reasonable assumption to make here until
proven otherwise.

\section{Current (``no freeze'') modelling}

The example presented in \textit{VolleyballMag} as the situation which will be
improved by the new proposal is the situation where a team is at set/match
point, up by just one point and receiving - i.e. a scoreline of 19-20, or 14-15
in the final set. Given that there are only two outcomes of each point (lets
can be ignored as they do not change the match state), and with the assumptions
made above, we can produce a state transition diagram for the end of a set:

TODO insert diagram here

Each state is labelled with (Team A score-Team B score, serving team) as
opposed to the traditional ``serving team first'' notation. The important
simplification made here is that there are only 4 states once the score
reaches 20-20: a level scoreline with either team serving, or a team leading by
one and serving.

As the assumptions made above allow us to model this as a Markov chain, we can
produce the transition matrix:

$$
\begin{bmatrix}
  0 & p & 0 & 0 & 0 & (1-p) & 0 \\
  0 & 0 & (1-p) & p & 0 & 0 & 0 \\
  0 & (1-p) & 0 & 0 & 0 & p & 0 \\
  0 & 0 & 0 & 0 & (1-p) & 0 & p \\
  0 & 0 & p & (1-p) & 0 & 0 & 0 \\
  0 & 0 & 0 & 0 & 0 & 1 & 0 \\
  0 & 0 & 0 & 0 & 0 & 0 & 1 \\
\end{bmatrix}
$$

and from there the set of equations to calculate the absorbtion probability of state 5
(team A wins) from each initial state, $a_n$:

\begin{eqnarray*}
  a_0 & = & pa_1 + (1-p)a_5 \\
  a_1 & = & (1-p)a_2 + pa_3 \\
  a_2 & = & (1-p)a_1 + pa_5 \\
  a_3 & = & (1-p)a_4 + pa_6 \\
  a_4 & = & pa_2 + (1-p)a_3 \\
  a_5 & = & a_5 \\
  a_6 & = & a_6
\end{eqnarray*}

Trivially, $a_5 = 1$ and $a_6 = 0$ so these reduce to:

\begin{eqnarray*}
  a_0 & = & pa_1 + (1-p) \\
  a_1 & = & (1-p)a_2 + pa_3 \\
  a_2 & = & (1-p)a_1 + p \\
  a_3 & = & (1-p)a_4 \\
  a_4 & = & pa_2 + (1-p)a_3
\end{eqnarray*}

We now solve these five simultaneous equations for $a_0$ to $a_4$:

\begin{eqnarray*}
  a_0 & = & \frac{3 \left(p - 1\right)}{2 p - 3} \\
  a_1 & = & \frac{2 \left(p - 1\right)}{2 p - 3} \\
  a_2 & = & \frac{p - 2}{2 p - 3} \\
  a_3 & = & \frac{p - 1}{2 p - 3} \\
  a_4 & = & - \frac{1}{2 p - 3}
\end{eqnarray*}

Of these, we particularly care about $a_0 = \frac{3 \left(p - 1\right)}{2 p -
3}$, the chance that team A wins given the 19-20 situation in
\textit{VolleyballMag}.

\section{Freeze modelling}

We can now repeat all the above for the ``freeze'' situation. The state
transition diagram is as follows:

TODO insert diagram here

and the transition matrix as follows:

$$
\begin{bmatrix}
  0 & (1-p) & p & 0 & 0 & 0 \\
  (1-p) & 0 & 0 & 0 & p & 0 \\
  0 & 0 & 0 & (1-p) & 0 & p \\
  0 & 0 & (1-p) & 0 & p & 0 \\
  0 & 0 & 0 & 0 & 1 & 0 \\
  0 & 0 & 0 & 0 & 0 & 1
\end{bmatrix}
$$

the absorption probabilities of state 4 as follows:

\begin{eqnarray*}
  b_0 & = & (1-p)b_1 + pb_2 \\
  b_1 & = & (1-p)b_0 + pb_4 \\
  b_2 & = & (1-p)b_3 + pb_5 \\
  b_3 & = & (1-p)b_2 + pb_4 \\
  b_4 & = & b_4 \\
  b_5 & = & b_5
\end{eqnarray*}

In this case, $b_4 = 1$ and $b_5 = 0$ so these reduce to:

\begin{eqnarray*}
  b_0 & = & (1-p)b_1 + pb_2 \\
  b_1 & = & (1-p)b_0 + p \\
  b_2 & = & (1-p)b_3 \\
  b_3 & = & (1-p)b_2 + p \\
\end{eqnarray*}

and we can solve these to get:

\begin{eqnarray*}
  b_0 & = & \frac{p^{2} - 4 p + 3}{p^{2} - 4 p + 4} \\
  b_1 & = & \frac{p^{2} - 3 p + 3}{p^{2} - 4 p + 4} \\
  b_2 & = & \frac{p - 1}{p - 2} \\
  b_3 & = & - \frac{1}{p - 2}
\end{eqnarray*}

Again here we are particularly interested in $b_0 = \frac{p^{2} - 4 p +
3}{p^{2} - 4 p + 4}$.

\section{Results}

From the main results of the previous two sections, we have

\begin{eqnarray*}
  a_0 & = & \frac{3 \left(p - 1\right)}{2 p - 3} \\
  b_0 & = & \frac{p^{2} - 4 p + 3}{p^{2} - 4 p + 4}
\end{eqnarray*}

To get a value for $p$, we can use the information from the
\textit{VolleyballMag} article:

\begin{quote}
  the team receiving serve will score 70-80\% of the time
\end{quote}

to estimate $p = 0.75$. Plugging this into the equations above gives us:

\begin{eqnarray*}
  a_0 & = & 0.9 \\
  b_0 & \approx & 0.67
\end{eqnarray*}

so this proposal would reduce the chance of team A winning by around 23\%, a
significant amount.

\begin{enumerate}
  \item \url{https://volleyballmag.com/beach-rule-changes-083020/}
\end{enumerate}

\end{document}
